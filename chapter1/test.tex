\documentclass{easybook}
\usepackage{mathtools}
\usetikzlibrary{angles,quotes}
\begin{document}
\begin{enumerate}
    \item 小孔成像
    \item \begin{eqnarray*}
              c'=&\frac{c}{n}\\
              c_{水}=&\frac{c}{1.33}\\
              c_{冕牌玻璃}=&\frac{c}{1.51}\\
              c_{火石玻璃}=&\frac{c}{1.65}\\
              c_{加拿大树胶}=&\frac{c}{1.526}\\
              c_{金刚石}=&\frac{c}{2.417}
          \end{eqnarray*}
    \item 求像距$s^{'}_{1}$
          \begin{eqnarray*}
              & \frac{h}{s}=\frac{h^{'}_{1}}{s^{'}_{1}}=\frac{h^{'}_{2}}{s^{'}_{2}}\\
              & h^{'}_{1} = 60mm,h^{'}_{2} = 70mm, s^{'}_{2} = s^{'}_{1}+50mm\\
              & s^{'}_{1} = \frac{h^{'}_{1}\times s^{'}_{2}}{h^{'}_{2}} = \frac{60mm\times (s^{'}_{1}+50mm)}{70mm}\\
              & s^{'}_{1} = 300mm
          \end{eqnarray*}
    \item 因为全反射的条件是入射角大于临界角,所以
          \begin{eqnarray*}
              & n \sin(\theta) = n' \sin(\theta')\\
              & n = 1.5,n' = 1.0,\theta' = \frac{\pi}{2}\\
              & \sin(\theta) = \frac{2}{3}\\
              & w_{max} = w + 2h\tan(\theta) = 1mm+2\times 200mm\tan(\arcsin(\frac{2}{3})) = 358.77mm
          \end{eqnarray*}
    \item 光疏到光密的界面,无法全反射
    \item 设入射角为$\theta$,入射介质折射率为$n$,出射介质折射率为$n'$,则有
          \begin{equation*}
              n \sin(\theta) = n' \sin(\theta')
          \end{equation*}
          则出射平板的角也为$\theta$,可知入射光线与出射光线平行。
    \item \begin{equation*}
              n \sin(\alpha) = \sin(\delta)
          \end{equation*}
    \item 光纤内需全反射,
          \begin{eqnarray*}
              & \frac{n_2}{n_1} = \sin (\theta_{min}) \\
              & n_0 \sin(I_1) = n_1 \sin (\theta_{min}) = n_2
          \end{eqnarray*}
    \item \begin{eqnarray*}
              & \sin(\theta_0) = n \sin(\theta_{0}^{'}) \\
              & \frac{\pi}{4} = \theta_{0}^{'} + \theta_1 \\
              & n \sin(\theta_1) \geq 1 \\
              &   0 \leq \theta_0  \leq 0.099
          \end{eqnarray*}
    \item 费马证明折射

          \begin{tikzpicture}
              \draw (-4,0) -- (4,0) node[right] {界面};
              \node [above] at (0,2) {介质1($n$)};
              \draw [thick] (-3,2) -- (0,0);
              \draw [thick] (0,-1) -- (0,1);
              \path (-3,2) coordinate (A) (0,0) coordinate (O) (0,1) coordinate (N);
              \pic [draw, angle radius=8mm, "$\theta$"] at (O) {angle = N--O--A};
              \path (0,0) coordinate (O) (2,-2) coordinate (B) (0,-1) coordinate (M);
              \pic [draw, angle radius=8mm, "$\theta'$"] at (O) {angle = M--O--B};
              \draw [thick,->] (0,0) -- (2,-2);
              \draw [thick,->] (0,0) -- (2,-2);
              \draw [dashed] (-3,2) -- (-3,0) node[midway,left] {$h$};
              \draw [dashed] (2,-2) -- (2,0) node[midway,right] {$h'$};
              % ISO-style dimension from (-3,0) to (2,0) labeled L
              \draw (-3,0) -- ++(0,-0.8); % left extension line
              \draw (2,0)  -- ++(0,-0.8); % right extension line
              \draw (0,0)  -- ++(0,-0.8); % right extension line
              \draw[<->, >=stealth, line cap=butt] (-3,-0.6) -- node[below=2pt] {$L$} (2,-0.6);
              \draw[<->, >=stealth, line cap=butt] (-3,-0.3) -- node[below=2pt] {$x$} (0,-0.3);
          \end{tikzpicture}

          根据几何关系,有光程
          \begin{eqnarray*}
              & S_1 = n\sqrt{h^2 + x^2} \\
              & S_2 = n'\sqrt{h^{\prime2} + (L-x)^2} \\
              & S = S_1 + S_2 = n\sqrt{h^2 + x^2} + n'\sqrt{h^{\prime2} + (L-x)^2} \\
          \end{eqnarray*}

          由费马原理,$S$取极值时,光线实际走过的路径。对$x$求导,有
          \begin{equation*}
              \frac{dS}{dx} = \frac{n x}{\sqrt{h^2 + x^2}} - \frac{n' (L-x)}{\sqrt{h^{\prime2} + (L-x)^2}} = 0
          \end{equation*}

          即
          \begin{eqnarray*}
              & \frac{n x}{\sqrt{h^2 + x^2}} = \frac{n' (L-x)}{\sqrt{h^{\prime2} + (L-x)^2}} \\
              & n \sin(\theta) = n \frac{x}{\sqrt{h^2 + x^2}} \\
              & n' \sin(\theta') = n' \frac{L-x}{\sqrt{h^{\prime 2} + (L-x)^2}} \\
              & n \sin(\theta) = n' \sin(\theta')
          \end{eqnarray*}
    \item 考虑光程差存在于反射段则
          \begin{eqnarray*}
              & \tan\left(\theta\right) = \frac{dy}{dx} \\
              & f = y + \frac{x}{\tan(\theta)} \\
              & \frac{df}{dx} = 0 = \frac{1+y^{\prime2}}{2y^\prime}-
              \frac{xy^{\prime\prime}(1+y^{\prime2})}{2y^{\prime2}} \\
          \end{eqnarray*}
          化简为
            \begin{equation*}
                x y^{\prime\prime} - y^{\prime2} = 0
            \end{equation*}
            解得
            \begin{equation*}
                y = C_1 x^2 + C_2
            \end{equation*}
          即抛物线
    \item 考虑光程和
    \begin{equation*}
        c = n \sqrt{(x-l)^2 + y^2} + n^{\prime} \sqrt{(x-l^{\prime})^2 + y^2}
    \end{equation*}
    即笛卡尔卵形线
    \item \begin{eqnarray*}
                & \beta = \frac{y'}{y} = \frac{nl^{\prime}}{n^{\prime}l} \\
                & n_k^\prime = n_{k+1} \\
                & \beta = \frac{y_1^\prime}{y_1} \cdot \frac{y_2^\prime}{y_2} = \frac{n_1 l^{\prime}_1}{n^{\prime}_1 l_1} \cdot \frac{n_2 l^{\prime}_2}{n^{\prime}_2 l_2} \\
                & \beta = \frac{n_1 l^{\prime}_1 l^{\prime}_2}{n^{\prime}_2 l_1 l_2} 
    \end{eqnarray*}
    \item 画图可知
    \begin{equation}
        2\theta = \theta^{\prime}
    \end{equation}
    所以考虑小角即$\sin(\theta) \approx \theta$,有
    \begin{eqnarray*}
        & n \sin(\theta) = \sin(\theta^{\prime}) \\
        & n \theta = 2\theta \\
        & n = 2
    \end{eqnarray*}
    \item 入射和出射光线平行
    \item \begin{eqnarray*}
        & l_1 = -\infty,  r = 30mm , n_1^\prime = 1.5, n_1 = 1.0 \\
        & \frac{n_1^\prime}{l_1^\prime} - \frac{n_1}{l_1} = \frac{n_1^\prime - n_1}{r} \\
        & l_1^\prime = 90mm \\
        & l_2 = l_1^\prime - 2r = 30mm,n_2 = n_1^\prime,n_2^\prime = n_1 \\
        & \frac{n_2^\prime}{l_2^\prime} - \frac{n_2}{l_2} = \frac{n_2^\prime - n_2}{-r} \\
        & l_2^\prime = 15mm \\
        & \beta = \frac{l_1^\prime l_2^\prime}{l_1 l_2} < 0 
    \end{eqnarray*}
    为实像
    \begin{eqnarray*}
        & l = -\infty,  r = 30mm  \\
        & \frac{1}{l^\prime} + \frac{1}{l} = \frac{2}{r} \\
        & l^\prime = 15mm
    \end{eqnarray*}
    为虚像

    由第一个公式可知,$l_2 = 30mm$
    \begin{eqnarray*}
        & \frac{1}{l_2^\prime} + \frac{1}{l_2} = \frac{2}{-r} \\
        & l_2^\prime = -10mm \\
        & l_3 = l_2^\prime + 2r = 50mm,n_3 = n_2 = 1.5,n_3^\prime = n_2^\prime = 1 \\
        & \frac{n_3^\prime}{l_3^\prime} - \frac{n_3}{l_3} = \frac{n_3^\prime - n_3}{r} \\
        & l_3^\prime = 75mm \\
    \end{eqnarray*}
    \item
    \begin{eqnarray*}
        &\beta = \frac{nl^{\prime}}{n^{\prime}l}\\
        & r = 150mm,n =1 ,n^{\prime} = 1.5\\
        & n\left(\frac{1}{r}-\frac{1}{l}\right) = n^{\prime}\left(\frac{1}{r}-\frac{1}{l^{\prime}}\right)\\
        & \beta = \frac{2r}{2r+l}\\
        & l = -\infty, l^{\prime} = 450mm,\beta = 0\\
        & l = -1000mm,l^{\prime} = 642.8571mm,\beta = 0.23\\
        & l = -100mm,l^{\prime} = -225mm,\beta = 0.75\\
        & l = 0mm,l^{\prime} = 0mm,\beta = 1\\
        & l = 150mm,l^{\prime} = 150 mm ,\beta = 2\\
        & l = 200mm,l^{\prime} = 180 mm ,\beta = 3
    \end{eqnarray*}
    \item 中心气泡
    \begin{eqnarray*}
        & l = r = -200mm,n =1.5 ,n^{\prime} = 1 \\
        & \frac{n^{\prime}}{l^{\prime}} - \frac{n}{l} = \frac{n^{\prime} - n}{r} \\
        & l^{\prime} = -200mm
    \end{eqnarray*}
    1/2半径气泡
    \begin{eqnarray*}
        & l = -300mm ,r = -200mm,n =1.5 ,n^{\prime} = 1 \\
        & \frac{n^{\prime}}{l^{\prime}} - \frac{n}{l} = \frac{n^{\prime} - n}{r} \\
        & l^{\prime} = -400mm
    \end{eqnarray*}
    换一边观察
    \begin{eqnarray*}
        & l = -100mm,r=-200mm,n =1.5 ,n^{\prime} = 1 \\
        & l^{\prime} = -80mm
    \end{eqnarray*}
    \item \begin{eqnarray*}
        & l_1 = -\infty,r_1 = 100mm,r_2 = \infty,n_1 = 1,n_1^{\prime} = n_2 = 1.5,n_2^{\prime} = 1 \\
        & \frac{n_1^{\prime}}{l_1^{\prime}} - \frac{n_1}{l_1} = \frac{n_1^{\prime} - n_1}{r_1} = \frac{n_1^{\prime}}{l_1^{\prime}} \\
        & l_1^{\prime} = 300mm,l_2 = l_1^{\prime} - d = 0mm\\
        & l_2^{\prime} = 0mm\\
    \end{eqnarray*}
    在第二面的十字线上
    \begin{eqnarray*}
        & l_1 = -d = -300mm,r =-100mm,n = 1.5 ,n^{\prime} = 1 \\
        & \frac{n^{\prime}}{l^{\prime}} - \frac{n}{l} = \frac{n^{\prime} - n}{r} \\
        & l^{\prime} = \infty\\
    \end{eqnarray*}
    $h=10mm$时不能认为是高斯成像
    \begin{eqnarray*}
        & \sin(\theta_1) = \frac{10mm}{300mm}\\
        & n_1 \sin(\theta_1) = n_1^{\prime} \sin(\theta_1^{\prime})\\
        & \frac{\sin(\theta_1^{\prime}-\theta_1) }{r} = \frac{\sin(\theta_1^{\prime})}{l_1^{\prime} - r}\\
        & \theta_2 = \theta_1^{\prime} - \theta_1 \\
        & n_1^{\prime} \sin(\theta_2) = n_1 \sin(\theta_2^{\prime})\\
        & l_2 = l_1^{\prime} - d \\
        & h_2 = l_2 \tan(-\theta_2) \\
        & l_2^{\prime} = \frac{h_2}{\tan(-\theta_2^{\prime})}
    \end{eqnarray*}
    \item \begin{eqnarray*}
        & r = -100mm,\beta = -\frac{l^{\prime}}{l},\frac{1}{l^{\prime}} + \frac{1}{l} = \frac{2}{r}\\
        & \beta  = 0,l = -\infty , l^{\prime} = -50mm\\
        & \beta  = -0.1,l = -550mm,l^{\prime} = -55mm\\
        & \beta  = -0.2,l = -300mm,l^{\prime} = -60mm\\
        & \beta  = -1,l = -100mm,l^{\prime} = -100mm\\
        & \beta = 1,l = 0mm,l^{\prime} = 0mm\\
        & \beta  = 5,l = -40mm,l^{\prime} = 200mm\\
        & \beta  = 10,l = -40mm,l^{\prime} = 450mm\\
        & \beta = \infty,l = -50mm,l^{\prime} = \infty\\
    \end{eqnarray*}
    \item $\beta <0$ 时虚实相同,$\beta >0$ 时虚实相反
    
    \begin{eqnarray*}
        & \beta = -4,l = \frac{8}{5r},l^{\prime} = \frac{5}{2r}\\
        & \beta = 4,l = \frac{8}{3r},l^{\prime} = -\frac{3}{2r}\\
        & \beta = -\frac{1}{4},l = \frac{5}{2r},l^{\prime} = \frac{8}{5r}\\
        & \beta = \frac{1}{4},l = -\frac{3}{2r},l^{\prime} = \frac{3}{8r}\\
    \end{eqnarray*}

    \item 考虑几何关系,只有入射第一面的像成像于反射面上,出射光线与入射光线平行
    \begin{eqnarray*}
        & l_1 = -\infty , l_1^{\prime}=2r \\
        & \frac{n}{l_1^{\prime}} - \frac{1}{l_1} = \frac{n-1}{r} \\
        & n = 2\\
        \end{eqnarray*}
    \item \begin{equation}
        \theta = (2n-1)\alpha
    \end{equation}
\end{enumerate}
\end{document}